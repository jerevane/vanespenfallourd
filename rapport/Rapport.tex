\documentclass[12pt,a4paper]{report}
\usepackage{graphicx}
\usepackage{pdfpages}
\usepackage{hyperref}
\usepackage{wrapfig}
\usepackage{lscape}
\usepackage{rotating}
\usepackage{epstopdf}
\usepackage[utf8]{inputenc}
\usepackage[cyr]{aeguill}
\usepackage[francais]{babel}
\graphicspath{ {images/} }
\hypersetup{
    colorlinks,
    citecolor=black,
    filecolor=black,
    linkcolor=black,
    urlcolor=black
}
\begin{document}
\includepdf{Titlepage}
\tableofcontents
\chapter{Objectif}
\section{Présentation générale}
Le jeu est bas\'{e} sur l'arch\'{e}type de Final Fantasy X, se distinguant par un système de combat au tour par tour ainsi que son embl\'{e}matique "sph\'{e}rier" correspondant à un arbre de talent apportant une libert\'{e} de personnalisation non n\'{e}gligeable. Voici un croquis du sphérier :

\includegraphics[width=0.80\textwidth]{Spherier.jpg}

A la diff\'{e}rence du v\'{e}ritable jeu, il n'y aura pas de d\'{e}placement libre en temps r\'{e}el; le gameplay se r\'{e}sume à un d\'{e}placement de points d'int\'{e}rêts en points d'int\'{e}rêts, poussé par une histoire typique de cet arch\'{e}type, où se déroulent diff\'{e}rents \'{e}vènements/combats.



\section{Règles du jeu}
Le coeur du jeu sont les deux personnages auquel le joueur (si il joue seul) a accès. Chacun commence avec une sp\'{e}cialisation particulière de mage et de guerrier choisissant directement de se diriger vers diff\'{e}rentes sp\'{e}cialisations. Via une carte du monde, le joueur avance de point en point et d\'{e}clenche diff\'{e}rentes rencontres. Les combats se d\'{e}roulent en tour par tour et si l'un des joueurs meurt, il revient à la vie avec 1 point de vie. La mort des deux joueurs entraine un retour au dernier point de sauvegarde automatique. Gagner un combat donne de l'exp\'{e}rience, qui donne des niveaux, qui permettent de progresser dans le sph\'{e}rier. 

Il existe \'{e}galement diff\'{e}rents objets permettant de rendre de la vie ou d'avoir d'autres effets. De plus, un système d'\'{e}quipement existe pour rendre le joueur plus fort au cours de l'aventure. En effet, un personnage possède des caract\'{e}ristiques (Force, Agilit\'{e}, Intelligence, Points de vie (HP) et Points de magie (MP)) qu'il peut am\'{e}liorer.

Le jeu se finit lorsque l'aventure est termin\'{e}e !
\section{Conception Logiciel}

Voici les packages de notre projet :

\textbf{Package state.} Package central qui gère l'\'{e}tat du jeu.

\textbf{Package engine.} Package qui modifie l'\'{e}tat de jeu en fonction de diff\'{e}rents inputs.

\textbf{Package ai.} Package qui gère le contrôle par l'ordinateur des monstres et, potentiellement, un personnage principal.

\textbf{Package server.} Package contenant la gestion de l'API du jeu, que ce soit par r\'{e}seau ou localement.

\textbf{Package instance.} Gère les différents contexte de jeu et leur rendu.

\textbf{Package client.} Package contenant les diff\'{e}rents traitements et commandes faites localement, sur la machine du joueur, afin de produire le comportement souhait\'{e} du jeu.

\textbf{Package entity.} Package contenant toutes les entit\'{e}s du jeu. Cela rassemble les Personnages Non Joueurs, les personnages principaux et les diff\'{e}rents ennemis.

\textbf{Package capacities.} Package contenant les diff\'{e}rents sorts et capacit\'{e}s utilis\'{e}es par les personnages principaux et les ennemis.

\begin{figure}
\caption{Diagramme des packages}
\includegraphics[width=0.80\textwidth]{Diagramme_packages.jpg}
\end{figure}

\chapter{Description et conception des \'{e}tats}
\section{Description des \'{e}tats}

Un \'{e}tat du jeu est form\'{e} par un ensemble d'\'{e}l\'{e}ments vivants et non vivants ainsi que la situation dont se trouve le joueur.

\subsection{Etat \'{e}l\'{e}ments vivants}

Les \'{e}l\'{e}ments vivants sont des entit\'{e}s ayant tous des caract\'{e}ristiques propres : santé max, santé actuelle, mana max, mana actuel, force, agilit\'{e} et intelligence.
On distingue deux types d'\'{e}l\'{e}ments vivants :

\textbf{Character.} Ce sont les h\'{e}ros du jeu. Ils pourront s'\'{e}quiper d'une arme et d'une protection ajoutant des bonus d'attaque et de d\'{e}fence. Ils \'{e}volueront grâce \`{a} un syst\`{e}me de gain d'exp\'{e}rience et de personalisation du joueur.  

\textbf{Monster.} Comme le nom l'indique, ce sont les monstres du jeu. Ils ne peuvent pas gagner d'exp\'{e}rience et leurs caract\'{e}ristiques sont g\'{e}n\'{e}r\'{e}es avec le niveau actuel des h\'{e}ros. Leurs comp\'{e}tences seront fix\'{e}es suivant le type du monstre (\'{e}l\'{e}mentaire, boss ect...). Ils ne portent pas d'\'{e}quipement. Seul les boss auront une capacit\'{e} sp\'{e}ciale (comme les h\'{e}ros) appel\'{e} "Overdrive".

\subsection{Etat \'{e}l\'{e}ments non vivants}

Les \'{e}l\'{e}ments non vivants sont au nombre de trois et ne portent aucune caract\'{e}ristiques communes. 

\textbf{Item.} Ce sont les objets utilisables par les h\'{e}ros. Ils peuvent changer leurs attributs (augmentation d'une caract\'{e}ristique ect...).

\textbf{SphereGrid.} C'est une table des compétences. Chaque niveaux suppl\'{e}mentaires permettra au character d'apprendre de nouvelles comp\'{e}tences et d'augmenter ces caract\'{e}ristiques. Il sera possible au joueur de choisir la personnalisation de son personnage car plusieurs table sont possible au cours du jeu.   

\textbf{Node.} Ce sont les points clefs du jeu. Les h\'{e}ros pourront se d\'{e}placer sur une carte de noeud en noeud. Chaque noeud comporte des \'{e}v\`{e}nements al\'{e}atoires et non al\'{e}atoires. Les \'{e}l\'{e}ments al\'{e}atoires sont des combats contre des monstres al\'{e}atoires tandis que les non al\'{e}atoires sont des \'{e}l\`{e}ments de l'histoire. Cela peut \^{e}tre un simple dialogue ou un combat contre un boss. Il est possible uniquement de ce d\'{e}placer au noeud suivant ou au noeud pr\'{e}c\'{e}dent. Si ce noeud a d\'{e}j\`{a} \'{e}t\'{e} visit\'{e}, l'\'{e}v\`{e}nement non al\'{e}atoire n'aura plus lieu.

\subsection{Situation du joueur}

Les situations possibles dans lequelles se trouve le joueur sont au nombre de quatre. Elles repr\'{e}sentent la ligne directive du jeu.

\textbf{D\'{e}placement sur un noeud.} Le joueur se d\'{e}place dans un nouvel endroit qui va g\'{e}n\'{e}rer des \'{e}v\`{e}nements. 

\textbf{Ev\`{e}nement al\'{e}atoire.} Cet \'{e}v\`{e}nement se traduit la plus part du temps par un combat contre des monstres al\'{e}atoires.

\textbf{Aubergiste.} En arrivant dans un nouveau lieu, le joueur a acc\`{e}s \`{a} un menu lui permettant d'acheter des objets utilisables en combat et de se pr\'{e}parer \`{a} l'\'{e}v\`{e}nement non al\'{e}atoire.

\textbf{Ev\'{e}nement non al\'{e}atoire.} Il d\'{e}pend de l'histoire du jeu.

\newpage

\section{Conception logiciel}

Dans cette section, nous expliciterons le diagrammes des classes pour les \'{e}tats pr\'{e}sent\'{e} en fin de chapitre. 

\textbf{Classes Element.} Cette classe et ses classes filles contiennent tout les \'{e}l\'{e}ments vivants du jeu. On distingue deux classes filles : Character et Monster. La classe Character possède des dépendances avec des classes comme SphereGrid et Item qui représentent l'\'{e}volution des personnages ainsi que les objets dont ils pourront faire l'usage. 

\textbf{Fabrique d'\'{e}l\'{e}ments.} Cette classe a pour but de rendre plus facile la construction d'\'{e}l\'{e}ments vivants, notamment des monstres provenant d'un \'{e}v\`{e}nement al\'{e}atoire (consid\'{e}r\'{e} comme non boss).

\textbf{Liste d'\'{e}l\'{e}ments.} Elle va contenir toute les informations sur l'ensemble des \'{e}l\'{e}ments pr\'{e}sent dans le jeu. 

\textbf{Classes Node.} Cette classe permet de faire le lien entre l'apparition d'\'{e}v\`{e}nements et les \'{e}l\'{e}ments. C'est une liste chain\'{e}e dont le d\'{e}placement est bidirectionnel. 

\textbf{Classe State.} Cette classe contiendra toutes les informations li\'{e}es au noeud o\`{u} se trouve le joueur et aux \'{e}l\'{e}ments vivants / non vivants encore pr\'{e}sent.

\textbf{Observer.} Cette classe permettra de relever les changements d'\'{e}tats du jeu et de le transmettre aux autres packages.

\begin{sidewaysfigure}[ht]
\caption{Diagramme des classes du package state}
\includegraphics[width=1.25\textwidth]{state.jpeg}
\end{sidewaysfigure}

\chapter{Rendu : Strat\'{e}gie et Conception}
\section{Strat\'{e}gie de rendu d'un état}
Notre jeu \'{e}tant à temps discret, le rendu d'un \'{e}tat sera assez simple. Le tout étant de distinguer entre un changement d'\'{e}tat qui n'entra\^{i}ne pas un lourd besoin de rendu (ie. pendant un combat) et un changement d'\'{e}tat demandant un changement de sprites (ie. changer de ville).

Pour se faire, nous introduisons le concept d'instance, offrant une classe par contexte (ie. Combat, Carte du monde, Menu d'intro...) g\'{e}rant la logique d'initialisation et les donn\'{e}es relatives des changments d'\'{e}tat du jeu. Le rendu de ces classes, utilisant les m\'{e}thodes de SFML, sera faite par les classes du package render.

\section{Conception logiciel}
Le diagramme des classes du package render se trouve en fin de cette partie. Nous expliquerons le r\^{o}le de chaque classe de ce diagramme.

\textbf{Renderer.} Cette classe contient toutes les informations communes entre chaque contexte. Elle est néanmoins abstraite et ses classes filles (WorldMapRender, FightRender ...) doivent ainsi red\'{e}finir ces m\'{e}thodes afin de diff\'{e}rencier correctement chaque contexte et chaque \'{e}tat du jeu.

\textbf{ElemSprite.} Cette classe poss\`{e}de les caract\'{e}ristiques des sprites des \'{e}l\'{e}ments du jeu. Nous pouvons, gr\^{a}ce \`{a} cette classe, personaliser et placer des \'{e}l\'{e}ments vivants et non vivants \`{a} chaque \'{e}tat du jeu. Elle n'est pour l'instant reli\'{e}e seulement \`{a} WorldMapRenderer car nous n'avons pas coder les autres instances.

\textbf{TextureSetter.} C'est dans cette classe que l'on va instancier toutes les textures des \'{e}l\'{e}ments vivants et non vivants du jeu. Ce tableau sera utilis\'{e} par les classes filles de Renderer.

\textbf{ClaudeUI, YouennUI.} Nous pr\'{e}voyons ses classes qui devront afficher un menu \`{a} l'utilisateur afin qu'il puisse choisir parmis les diff\'{e}rentes attaques disponible pour le personnage. 

\begin{sidewaysfigure}[ht]
\caption{Diagramme des classes du package render}
\includegraphics[width=1\textwidth]{render.jpeg}
\end{sidewaysfigure}
 
\section{Exemple de rendu}

Ecran d'acceuil du jeu avec la possibilité d'afficher une entité avec ses points de vie.

\begin{sidewaysfigure}[ht]
\caption{Ecran d'acceuil du jeu Final Fantastique}
\includegraphics[width=1\textwidth]{Ecran_acceuil_ff.png}
\end{sidewaysfigure}

\begin{sidewaysfigure}[ht]
\caption{Ecran d'acceuil avec l'elementaire de feu}
\includegraphics[width=1\textwidth]{Ecran_acceuil_fire_elemental.png}
\end{sidewaysfigure}

\chapter{R\`{e}gle de changement d'\'{e}tat}

Le jeu est \`{a} temps discret donc les changements d'\'{e}tats ne se feront uniquement lors d'appuis sur des touches du clavier par l'utilisateur. Chaque contexte offre des choix diff\'{e}rents \`{a} l'utilisateur. 

\section{Conception logiciel}

\textbf{Screen.} C'est la classe m\`{e}re du package. Elle fait le lien entre la classe Application, ses classes filles et le package state afin de lier l'\'{e}tat et le rendu.

\textbf{Application.} Cette classe fait le lien entre la biblioth\`{e}que SFML et les diff\'{e}rents \'{e}tats du jeu. En effet \`{a} chaque nouvel \'{e}tat nous aurons \`{a} instancier de nouveaux sprites.

\textbf{Info, Worldmap, Fight, Inn.} Les classes filles de Screen nous permette de mieux personaliser les contextes suivant les choix de l'utilisateur et sa position dans l'histoire du jeu.

\begin{sidewaysfigure}[ht]
\caption{Diagramme de classe du package instance}
\includegraphics[width=1\textwidth]{instance.jpeg}
\end{sidewaysfigure}

\chapter{IA}

\section{Comportement}

L'intelligence artificielle peut g\'{e}rer les personnages comme les monstres du jeux. Son comportement est dict\'{e} par la classe Rules (pr\'{e}sent dans le package engine) avec les deux variables bool\'{e}ennes AICharneeded et AIMonsterneeded.
Dans notre cas, l'intelligence artificielle doit choisir une action \`{a} effectuer sur une cible. Une premi\`{e}re i.a. devra choisir al\'{e}atoirement une action ainsi que sa cible, une seconde devra effectuer une action al\'{e}atoirement mais sur la meilleure cible et enfin une troisi\`{e}me devra choisir l'action la plus rentable sur la meilleure des cibles.
Pour cela nous utiliserons deux classes types : AI et ChoiceList.

\textbf{AI.} Cette classe contient toutes les informations dont \`{a} besoin une intelligence artificielle pour fonctionner. Les variables ChoiceTarget et ChoiceAction sont r\'{e}cup\'{e}r\'{e}es par l'engine qui devra actualiser l'\'{e}tat du jeu. Les sous classes RandomChoice et RandomGoodChoice sont les intelligences artificielles premi\`{e}res et secondes cit\'{e}es plus haut.

\textbf{ChoiceList.} Cette classe permet de cr\'{e}er un tableau avec les divers actions possibles sur n'importe quelles cibles. Ce tableau sera plus moins restreint suivant les besoins de l'intelligence artificielle utilis\'{e}e. Par exemple, l'i.a. RandomChoice aura besoin de toutes les combinaisons possibles d'action et de cible alors que RandomGoodChoice fera un premier tri sur les cibles suivant l'action execut\'{e}e. La restriction du tableau suivant les diff\'{e}rentes cibles possibles pour une action se base sur des paramètres logiques. Ainsi les actions d\'{e}fensives ne pourront pas \^{e}tre lanc\'{e}es sur des ennemies. 

\begin{sidewaysfigure}[ht]
\caption{Diagramme de classe du package ai}
\includegraphics[width=1\textwidth]{ai.jpeg}
\end{sidewaysfigure}

\end{document}
